
% Default to the notebook output style

    


% Inherit from the specified cell style.




    
\documentclass[11pt]{article}

    
    
    \usepackage[T1]{fontenc}
    % Nicer default font (+ math font) than Computer Modern for most use cases
    \usepackage{mathpazo}

    % Basic figure setup, for now with no caption control since it's done
    % automatically by Pandoc (which extracts ![](path) syntax from Markdown).
    \usepackage{graphicx}
    % We will generate all images so they have a width \maxwidth. This means
    % that they will get their normal width if they fit onto the page, but
    % are scaled down if they would overflow the margins.
    \makeatletter
    \def\maxwidth{\ifdim\Gin@nat@width>\linewidth\linewidth
    \else\Gin@nat@width\fi}
    \makeatother
    \let\Oldincludegraphics\includegraphics
    % Set max figure width to be 80% of text width, for now hardcoded.
    \renewcommand{\includegraphics}[1]{\Oldincludegraphics[width=.8\maxwidth]{#1}}
    % Ensure that by default, figures have no caption (until we provide a
    % proper Figure object with a Caption API and a way to capture that
    % in the conversion process - todo).
    \usepackage{caption}
    \DeclareCaptionLabelFormat{nolabel}{}
    \captionsetup{labelformat=nolabel}

    \usepackage{adjustbox} % Used to constrain images to a maximum size 
    \usepackage{xcolor} % Allow colors to be defined
    \usepackage{enumerate} % Needed for markdown enumerations to work
    \usepackage{geometry} % Used to adjust the document margins
    \usepackage{amsmath} % Equations
    \usepackage{amssymb} % Equations
    \usepackage{textcomp} % defines textquotesingle
    % Hack from http://tex.stackexchange.com/a/47451/13684:
    \AtBeginDocument{%
        \def\PYZsq{\textquotesingle}% Upright quotes in Pygmentized code
    }
    \usepackage{upquote} % Upright quotes for verbatim code
    \usepackage{eurosym} % defines \euro
    \usepackage[mathletters]{ucs} % Extended unicode (utf-8) support
    \usepackage[utf8x]{inputenc} % Allow utf-8 characters in the tex document
    \usepackage{fancyvrb} % verbatim replacement that allows latex
    \usepackage{grffile} % extends the file name processing of package graphics 
                         % to support a larger range 
    % The hyperref package gives us a pdf with properly built
    % internal navigation ('pdf bookmarks' for the table of contents,
    % internal cross-reference links, web links for URLs, etc.)
    \usepackage{hyperref}
    \usepackage{longtable} % longtable support required by pandoc >1.10
    \usepackage{booktabs}  % table support for pandoc > 1.12.2
    \usepackage[inline]{enumitem} % IRkernel/repr support (it uses the enumerate* environment)
    \usepackage[normalem]{ulem} % ulem is needed to support strikethroughs (\sout)
                                % normalem makes italics be italics, not underlines
    

    
    
    % Colors for the hyperref package
    \definecolor{urlcolor}{rgb}{0,.145,.698}
    \definecolor{linkcolor}{rgb}{.71,0.21,0.01}
    \definecolor{citecolor}{rgb}{.12,.54,.11}

    % ANSI colors
    \definecolor{ansi-black}{HTML}{3E424D}
    \definecolor{ansi-black-intense}{HTML}{282C36}
    \definecolor{ansi-red}{HTML}{E75C58}
    \definecolor{ansi-red-intense}{HTML}{B22B31}
    \definecolor{ansi-green}{HTML}{00A250}
    \definecolor{ansi-green-intense}{HTML}{007427}
    \definecolor{ansi-yellow}{HTML}{DDB62B}
    \definecolor{ansi-yellow-intense}{HTML}{B27D12}
    \definecolor{ansi-blue}{HTML}{208FFB}
    \definecolor{ansi-blue-intense}{HTML}{0065CA}
    \definecolor{ansi-magenta}{HTML}{D160C4}
    \definecolor{ansi-magenta-intense}{HTML}{A03196}
    \definecolor{ansi-cyan}{HTML}{60C6C8}
    \definecolor{ansi-cyan-intense}{HTML}{258F8F}
    \definecolor{ansi-white}{HTML}{C5C1B4}
    \definecolor{ansi-white-intense}{HTML}{A1A6B2}

    % commands and environments needed by pandoc snippets
    % extracted from the output of `pandoc -s`
    \providecommand{\tightlist}{%
      \setlength{\itemsep}{0pt}\setlength{\parskip}{0pt}}
    \DefineVerbatimEnvironment{Highlighting}{Verbatim}{commandchars=\\\{\}}
    % Add ',fontsize=\small' for more characters per line
    \newenvironment{Shaded}{}{}
    \newcommand{\KeywordTok}[1]{\textcolor[rgb]{0.00,0.44,0.13}{\textbf{{#1}}}}
    \newcommand{\DataTypeTok}[1]{\textcolor[rgb]{0.56,0.13,0.00}{{#1}}}
    \newcommand{\DecValTok}[1]{\textcolor[rgb]{0.25,0.63,0.44}{{#1}}}
    \newcommand{\BaseNTok}[1]{\textcolor[rgb]{0.25,0.63,0.44}{{#1}}}
    \newcommand{\FloatTok}[1]{\textcolor[rgb]{0.25,0.63,0.44}{{#1}}}
    \newcommand{\CharTok}[1]{\textcolor[rgb]{0.25,0.44,0.63}{{#1}}}
    \newcommand{\StringTok}[1]{\textcolor[rgb]{0.25,0.44,0.63}{{#1}}}
    \newcommand{\CommentTok}[1]{\textcolor[rgb]{0.38,0.63,0.69}{\textit{{#1}}}}
    \newcommand{\OtherTok}[1]{\textcolor[rgb]{0.00,0.44,0.13}{{#1}}}
    \newcommand{\AlertTok}[1]{\textcolor[rgb]{1.00,0.00,0.00}{\textbf{{#1}}}}
    \newcommand{\FunctionTok}[1]{\textcolor[rgb]{0.02,0.16,0.49}{{#1}}}
    \newcommand{\RegionMarkerTok}[1]{{#1}}
    \newcommand{\ErrorTok}[1]{\textcolor[rgb]{1.00,0.00,0.00}{\textbf{{#1}}}}
    \newcommand{\NormalTok}[1]{{#1}}
    
    % Additional commands for more recent versions of Pandoc
    \newcommand{\ConstantTok}[1]{\textcolor[rgb]{0.53,0.00,0.00}{{#1}}}
    \newcommand{\SpecialCharTok}[1]{\textcolor[rgb]{0.25,0.44,0.63}{{#1}}}
    \newcommand{\VerbatimStringTok}[1]{\textcolor[rgb]{0.25,0.44,0.63}{{#1}}}
    \newcommand{\SpecialStringTok}[1]{\textcolor[rgb]{0.73,0.40,0.53}{{#1}}}
    \newcommand{\ImportTok}[1]{{#1}}
    \newcommand{\DocumentationTok}[1]{\textcolor[rgb]{0.73,0.13,0.13}{\textit{{#1}}}}
    \newcommand{\AnnotationTok}[1]{\textcolor[rgb]{0.38,0.63,0.69}{\textbf{\textit{{#1}}}}}
    \newcommand{\CommentVarTok}[1]{\textcolor[rgb]{0.38,0.63,0.69}{\textbf{\textit{{#1}}}}}
    \newcommand{\VariableTok}[1]{\textcolor[rgb]{0.10,0.09,0.49}{{#1}}}
    \newcommand{\ControlFlowTok}[1]{\textcolor[rgb]{0.00,0.44,0.13}{\textbf{{#1}}}}
    \newcommand{\OperatorTok}[1]{\textcolor[rgb]{0.40,0.40,0.40}{{#1}}}
    \newcommand{\BuiltInTok}[1]{{#1}}
    \newcommand{\ExtensionTok}[1]{{#1}}
    \newcommand{\PreprocessorTok}[1]{\textcolor[rgb]{0.74,0.48,0.00}{{#1}}}
    \newcommand{\AttributeTok}[1]{\textcolor[rgb]{0.49,0.56,0.16}{{#1}}}
    \newcommand{\InformationTok}[1]{\textcolor[rgb]{0.38,0.63,0.69}{\textbf{\textit{{#1}}}}}
    \newcommand{\WarningTok}[1]{\textcolor[rgb]{0.38,0.63,0.69}{\textbf{\textit{{#1}}}}}
    
    
    % Define a nice break command that doesn't care if a line doesn't already
    % exist.
    \def\br{\hspace*{\fill} \\* }
    % Math Jax compatability definitions
    \def\gt{>}
    \def\lt{<}
    % Document parameters
    \title{PROV}
    
    
    

    % Pygments definitions
    
\makeatletter
\def\PY@reset{\let\PY@it=\relax \let\PY@bf=\relax%
    \let\PY@ul=\relax \let\PY@tc=\relax%
    \let\PY@bc=\relax \let\PY@ff=\relax}
\def\PY@tok#1{\csname PY@tok@#1\endcsname}
\def\PY@toks#1+{\ifx\relax#1\empty\else%
    \PY@tok{#1}\expandafter\PY@toks\fi}
\def\PY@do#1{\PY@bc{\PY@tc{\PY@ul{%
    \PY@it{\PY@bf{\PY@ff{#1}}}}}}}
\def\PY#1#2{\PY@reset\PY@toks#1+\relax+\PY@do{#2}}

\expandafter\def\csname PY@tok@w\endcsname{\def\PY@tc##1{\textcolor[rgb]{0.73,0.73,0.73}{##1}}}
\expandafter\def\csname PY@tok@c\endcsname{\let\PY@it=\textit\def\PY@tc##1{\textcolor[rgb]{0.25,0.50,0.50}{##1}}}
\expandafter\def\csname PY@tok@cp\endcsname{\def\PY@tc##1{\textcolor[rgb]{0.74,0.48,0.00}{##1}}}
\expandafter\def\csname PY@tok@k\endcsname{\let\PY@bf=\textbf\def\PY@tc##1{\textcolor[rgb]{0.00,0.50,0.00}{##1}}}
\expandafter\def\csname PY@tok@kp\endcsname{\def\PY@tc##1{\textcolor[rgb]{0.00,0.50,0.00}{##1}}}
\expandafter\def\csname PY@tok@kt\endcsname{\def\PY@tc##1{\textcolor[rgb]{0.69,0.00,0.25}{##1}}}
\expandafter\def\csname PY@tok@o\endcsname{\def\PY@tc##1{\textcolor[rgb]{0.40,0.40,0.40}{##1}}}
\expandafter\def\csname PY@tok@ow\endcsname{\let\PY@bf=\textbf\def\PY@tc##1{\textcolor[rgb]{0.67,0.13,1.00}{##1}}}
\expandafter\def\csname PY@tok@nb\endcsname{\def\PY@tc##1{\textcolor[rgb]{0.00,0.50,0.00}{##1}}}
\expandafter\def\csname PY@tok@nf\endcsname{\def\PY@tc##1{\textcolor[rgb]{0.00,0.00,1.00}{##1}}}
\expandafter\def\csname PY@tok@nc\endcsname{\let\PY@bf=\textbf\def\PY@tc##1{\textcolor[rgb]{0.00,0.00,1.00}{##1}}}
\expandafter\def\csname PY@tok@nn\endcsname{\let\PY@bf=\textbf\def\PY@tc##1{\textcolor[rgb]{0.00,0.00,1.00}{##1}}}
\expandafter\def\csname PY@tok@ne\endcsname{\let\PY@bf=\textbf\def\PY@tc##1{\textcolor[rgb]{0.82,0.25,0.23}{##1}}}
\expandafter\def\csname PY@tok@nv\endcsname{\def\PY@tc##1{\textcolor[rgb]{0.10,0.09,0.49}{##1}}}
\expandafter\def\csname PY@tok@no\endcsname{\def\PY@tc##1{\textcolor[rgb]{0.53,0.00,0.00}{##1}}}
\expandafter\def\csname PY@tok@nl\endcsname{\def\PY@tc##1{\textcolor[rgb]{0.63,0.63,0.00}{##1}}}
\expandafter\def\csname PY@tok@ni\endcsname{\let\PY@bf=\textbf\def\PY@tc##1{\textcolor[rgb]{0.60,0.60,0.60}{##1}}}
\expandafter\def\csname PY@tok@na\endcsname{\def\PY@tc##1{\textcolor[rgb]{0.49,0.56,0.16}{##1}}}
\expandafter\def\csname PY@tok@nt\endcsname{\let\PY@bf=\textbf\def\PY@tc##1{\textcolor[rgb]{0.00,0.50,0.00}{##1}}}
\expandafter\def\csname PY@tok@nd\endcsname{\def\PY@tc##1{\textcolor[rgb]{0.67,0.13,1.00}{##1}}}
\expandafter\def\csname PY@tok@s\endcsname{\def\PY@tc##1{\textcolor[rgb]{0.73,0.13,0.13}{##1}}}
\expandafter\def\csname PY@tok@sd\endcsname{\let\PY@it=\textit\def\PY@tc##1{\textcolor[rgb]{0.73,0.13,0.13}{##1}}}
\expandafter\def\csname PY@tok@si\endcsname{\let\PY@bf=\textbf\def\PY@tc##1{\textcolor[rgb]{0.73,0.40,0.53}{##1}}}
\expandafter\def\csname PY@tok@se\endcsname{\let\PY@bf=\textbf\def\PY@tc##1{\textcolor[rgb]{0.73,0.40,0.13}{##1}}}
\expandafter\def\csname PY@tok@sr\endcsname{\def\PY@tc##1{\textcolor[rgb]{0.73,0.40,0.53}{##1}}}
\expandafter\def\csname PY@tok@ss\endcsname{\def\PY@tc##1{\textcolor[rgb]{0.10,0.09,0.49}{##1}}}
\expandafter\def\csname PY@tok@sx\endcsname{\def\PY@tc##1{\textcolor[rgb]{0.00,0.50,0.00}{##1}}}
\expandafter\def\csname PY@tok@m\endcsname{\def\PY@tc##1{\textcolor[rgb]{0.40,0.40,0.40}{##1}}}
\expandafter\def\csname PY@tok@gh\endcsname{\let\PY@bf=\textbf\def\PY@tc##1{\textcolor[rgb]{0.00,0.00,0.50}{##1}}}
\expandafter\def\csname PY@tok@gu\endcsname{\let\PY@bf=\textbf\def\PY@tc##1{\textcolor[rgb]{0.50,0.00,0.50}{##1}}}
\expandafter\def\csname PY@tok@gd\endcsname{\def\PY@tc##1{\textcolor[rgb]{0.63,0.00,0.00}{##1}}}
\expandafter\def\csname PY@tok@gi\endcsname{\def\PY@tc##1{\textcolor[rgb]{0.00,0.63,0.00}{##1}}}
\expandafter\def\csname PY@tok@gr\endcsname{\def\PY@tc##1{\textcolor[rgb]{1.00,0.00,0.00}{##1}}}
\expandafter\def\csname PY@tok@ge\endcsname{\let\PY@it=\textit}
\expandafter\def\csname PY@tok@gs\endcsname{\let\PY@bf=\textbf}
\expandafter\def\csname PY@tok@gp\endcsname{\let\PY@bf=\textbf\def\PY@tc##1{\textcolor[rgb]{0.00,0.00,0.50}{##1}}}
\expandafter\def\csname PY@tok@go\endcsname{\def\PY@tc##1{\textcolor[rgb]{0.53,0.53,0.53}{##1}}}
\expandafter\def\csname PY@tok@gt\endcsname{\def\PY@tc##1{\textcolor[rgb]{0.00,0.27,0.87}{##1}}}
\expandafter\def\csname PY@tok@err\endcsname{\def\PY@bc##1{\setlength{\fboxsep}{0pt}\fcolorbox[rgb]{1.00,0.00,0.00}{1,1,1}{\strut ##1}}}
\expandafter\def\csname PY@tok@kc\endcsname{\let\PY@bf=\textbf\def\PY@tc##1{\textcolor[rgb]{0.00,0.50,0.00}{##1}}}
\expandafter\def\csname PY@tok@kd\endcsname{\let\PY@bf=\textbf\def\PY@tc##1{\textcolor[rgb]{0.00,0.50,0.00}{##1}}}
\expandafter\def\csname PY@tok@kn\endcsname{\let\PY@bf=\textbf\def\PY@tc##1{\textcolor[rgb]{0.00,0.50,0.00}{##1}}}
\expandafter\def\csname PY@tok@kr\endcsname{\let\PY@bf=\textbf\def\PY@tc##1{\textcolor[rgb]{0.00,0.50,0.00}{##1}}}
\expandafter\def\csname PY@tok@bp\endcsname{\def\PY@tc##1{\textcolor[rgb]{0.00,0.50,0.00}{##1}}}
\expandafter\def\csname PY@tok@fm\endcsname{\def\PY@tc##1{\textcolor[rgb]{0.00,0.00,1.00}{##1}}}
\expandafter\def\csname PY@tok@vc\endcsname{\def\PY@tc##1{\textcolor[rgb]{0.10,0.09,0.49}{##1}}}
\expandafter\def\csname PY@tok@vg\endcsname{\def\PY@tc##1{\textcolor[rgb]{0.10,0.09,0.49}{##1}}}
\expandafter\def\csname PY@tok@vi\endcsname{\def\PY@tc##1{\textcolor[rgb]{0.10,0.09,0.49}{##1}}}
\expandafter\def\csname PY@tok@vm\endcsname{\def\PY@tc##1{\textcolor[rgb]{0.10,0.09,0.49}{##1}}}
\expandafter\def\csname PY@tok@sa\endcsname{\def\PY@tc##1{\textcolor[rgb]{0.73,0.13,0.13}{##1}}}
\expandafter\def\csname PY@tok@sb\endcsname{\def\PY@tc##1{\textcolor[rgb]{0.73,0.13,0.13}{##1}}}
\expandafter\def\csname PY@tok@sc\endcsname{\def\PY@tc##1{\textcolor[rgb]{0.73,0.13,0.13}{##1}}}
\expandafter\def\csname PY@tok@dl\endcsname{\def\PY@tc##1{\textcolor[rgb]{0.73,0.13,0.13}{##1}}}
\expandafter\def\csname PY@tok@s2\endcsname{\def\PY@tc##1{\textcolor[rgb]{0.73,0.13,0.13}{##1}}}
\expandafter\def\csname PY@tok@sh\endcsname{\def\PY@tc##1{\textcolor[rgb]{0.73,0.13,0.13}{##1}}}
\expandafter\def\csname PY@tok@s1\endcsname{\def\PY@tc##1{\textcolor[rgb]{0.73,0.13,0.13}{##1}}}
\expandafter\def\csname PY@tok@mb\endcsname{\def\PY@tc##1{\textcolor[rgb]{0.40,0.40,0.40}{##1}}}
\expandafter\def\csname PY@tok@mf\endcsname{\def\PY@tc##1{\textcolor[rgb]{0.40,0.40,0.40}{##1}}}
\expandafter\def\csname PY@tok@mh\endcsname{\def\PY@tc##1{\textcolor[rgb]{0.40,0.40,0.40}{##1}}}
\expandafter\def\csname PY@tok@mi\endcsname{\def\PY@tc##1{\textcolor[rgb]{0.40,0.40,0.40}{##1}}}
\expandafter\def\csname PY@tok@il\endcsname{\def\PY@tc##1{\textcolor[rgb]{0.40,0.40,0.40}{##1}}}
\expandafter\def\csname PY@tok@mo\endcsname{\def\PY@tc##1{\textcolor[rgb]{0.40,0.40,0.40}{##1}}}
\expandafter\def\csname PY@tok@ch\endcsname{\let\PY@it=\textit\def\PY@tc##1{\textcolor[rgb]{0.25,0.50,0.50}{##1}}}
\expandafter\def\csname PY@tok@cm\endcsname{\let\PY@it=\textit\def\PY@tc##1{\textcolor[rgb]{0.25,0.50,0.50}{##1}}}
\expandafter\def\csname PY@tok@cpf\endcsname{\let\PY@it=\textit\def\PY@tc##1{\textcolor[rgb]{0.25,0.50,0.50}{##1}}}
\expandafter\def\csname PY@tok@c1\endcsname{\let\PY@it=\textit\def\PY@tc##1{\textcolor[rgb]{0.25,0.50,0.50}{##1}}}
\expandafter\def\csname PY@tok@cs\endcsname{\let\PY@it=\textit\def\PY@tc##1{\textcolor[rgb]{0.25,0.50,0.50}{##1}}}

\def\PYZbs{\char`\\}
\def\PYZus{\char`\_}
\def\PYZob{\char`\{}
\def\PYZcb{\char`\}}
\def\PYZca{\char`\^}
\def\PYZam{\char`\&}
\def\PYZlt{\char`\<}
\def\PYZgt{\char`\>}
\def\PYZsh{\char`\#}
\def\PYZpc{\char`\%}
\def\PYZdl{\char`\$}
\def\PYZhy{\char`\-}
\def\PYZsq{\char`\'}
\def\PYZdq{\char`\"}
\def\PYZti{\char`\~}
% for compatibility with earlier versions
\def\PYZat{@}
\def\PYZlb{[}
\def\PYZrb{]}
\makeatother


    % Exact colors from NB
    \definecolor{incolor}{rgb}{0.0, 0.0, 0.5}
    \definecolor{outcolor}{rgb}{0.545, 0.0, 0.0}



    
    % Prevent overflowing lines due to hard-to-break entities
    \sloppy 
    % Setup hyperref package
    \hypersetup{
      breaklinks=true,  % so long urls are correctly broken across lines
      colorlinks=true,
      urlcolor=urlcolor,
      linkcolor=linkcolor,
      citecolor=citecolor,
      }
    % Slightly bigger margins than the latex defaults
    
    \geometry{verbose,tmargin=1in,bmargin=1in,lmargin=1in,rmargin=1in}
    
    

    \begin{document}
    
    
    \maketitle
    
    

    
    \section{PROV Service and Share Economy
{[}Beta{]}}\label{prov-service-and-share-economy-beta}

\subsection{1. Prerequisite}\label{prerequisite}

\subsubsection{1.1 Sustainable Symmetric
Key}\label{sustainable-symmetric-key}

By leverage the power of distributed ledger, the conventional secured
connection techniques such as HTTPS/SSL can be drastically simplified.

Say we have \(N_A\) and \(N_B\) with the public key and private key to
be (\(PU_A\),\(PV_A\)) and (\(PU_B\), \(PV_B\)) respectively, the below
steps are only required to be excuted only once for all.

\begin{enumerate}
\def\labelenumi{\arabic{enumi}.}
\tightlist
\item
  \(N_A\) generate a symmetric key \(K_S\) (a.k.a. \textbf{Sustainable
  Symmetric Key} or \textbf{SSK} ).
\item
  \(N_A\) uses asymmetric algorithm \(E\), \(PU_B\), metadata
  \(\lambda\) and signiture function \(Sig\) to generate the message
  \(M = Sig(E(PU_B, \lambda),PV_A)\) , then boardcasts \(M\) to
  blockchain.
\end{enumerate}

After that, \(N_A\) and \(N_B\) can use \(K_S\) to secure the traffic
for unlimited times, without asymmetric handshake, no matter the traffic
is onchain or offchain.

To reset SSK due to key leakage or version upgrade, \(N_A\) can simple
repeat the above steps. \(N_B\) always take the latest one.

\subsection{2. PROV Service: Provision, Representation, Orientation and
Verification}\label{prov-service-provision-representation-orientation-and-verification}

\subsubsection{2.1 Brief}\label{brief}

Any service, no matter the consumer and provider parties comes from
human community or computer network, can be incontestably processed only
when four \textbf{predefined} phases achieved accurate consensus:
Provision, Representation, Orientation, Verification.

\textbf{Provision}: Providers announce the their capacities of handling
the services with the rate they can offer, after rational estimation.

\textbf{Representation}: Consumer abstracts and refines the requirement
from its raw demond, comes up with the standard representation of the
task. A good representation should:

\begin{enumerate}
\def\labelenumi{\arabic{enumi}.}
\tightlist
\item
  avoid any functional or qualitative dispution caused by indistinction
  or obscuration.
\item
  expose enough and just enough information to support the provider's
  service, in order to maximize privacy protection and minimize the cost
  of processing.
\end{enumerate}

\textbf{Orientation}: Consumer compares the providers' offers, picks up
the ones with his best interest and assigns the service to them.

After orientation, the providers can untilize any resource and tactic to
accomplish the mission. In the service's point of view, there's no
interaction, no regulation, no even suggestion. In this way, providers
are couraged to maintain competitiveness and innovate to increase the
profit space.

\textbf{Verification}: After the providers believe the mission is
accomplished, an predefined verification should be performed. Similar as
Representation, verification should avoid any functional or qualitative
dispution caused by indistinction or obscuration. The evalution also
decides how to dispense the commission.

In a mature market model, the phases of P\emph{rovision, Representation}
and \emph{Evaluation} are standardized between participated parties,
leaving \emph{Orientation} to consumer. In the following chapters,
you'll see that in a distributed ledger network, it's better to get
\emph{Orientation} process standardized as well, in order to the on
chain traffic and protect the providers from Attacks.

A distributed consensus network is \textbf{PROV compatiable}, if

\begin{enumerate}
\def\labelenumi{\arabic{enumi}.}
\tightlist
\item
  every node \(k\) in this network has an asynmmetric key pair
  (\(PU_k, PV_k\)), where \(PU_k\) is the public key (address) and
  \(PV_k\) is the private key used for signiture.
\item
  There's a universial currency to measure the value of service. There's
  a way to calculate the balance of each account. Denote the balance as
  \(b_k\) for node \(k\).
\item
  there are \(n\) types registered services \{\(S_n\)\}, \(n \ge 1\).
  For any \(i \in [1, n]\), define \(S_i \equiv\)
  \{\(P_{S_i},R_{S_i}, O_{S_i}, V_{S_i} , PU_{a_i}, \vec{\psi_{S_i}}\)\},
  where \(P_{S_i},R_{S_i}, O_{S_i}\) and \$V\_\{S\_i\} \$ are the
  Provision, Representation, Orientation and Evaluation functions
  developed to support service \(S_i\). \(PU_{a_i}\) is \(S_i\)'s
  auther's address. \(\vec{\psi_{S_i}}\) is \(S_i\) specific parameters.
  We call \(S_i\) a \textbf{PROV Service}.
\item
  Suggest supports \emph{Sustainable Symmetric Key} introduced in
  \emph{Chapter III}, but not mandatory.
\end{enumerate}

\subsubsection{2.2 PROV Service life
cycle}\label{prov-service-life-cycle}

In \emph{PROV compatiable network}, A typical life cycle of instance
\(s \in S_i\) is described below:

\begin{enumerate}
\def\labelenumi{\arabic{enumi}.}
\item
  The service provider node \(p\) has resource \(\vec{\nu}\) (Say
  storage size, flops, etc.), it estimates it's capacity of running
  \(S_i\) by running
  \((\tilde{Cap}^{p}_{S_i}, \vec{\rho_{S_i}^p} ) = P_{S_i}(\vec{\nu})\).
  Here \(\tilde{Cap}^{p}_{S_i}\) is calculated for \(p\)'s reference.
  \(p\) boardcasts an provision message with it's capacity and rate,
  where \$\vec{\rho_{S_i}^p} \$ is service specific parameters. With
  \$Cap\textsuperscript{\{p\}\emph{\{Si\} \$(
  \(\le\tilde{Cap}^{p}_{S_i}\) ) along with rate \(Rate^{p}_{Si}\), the
  message is like:
  \$M}\{P\}(PU\_\{p\},Cap}\{p\}\emph{\{S\_i\},Rate\^{}\{p\}}\{S\_i\},
  \vec{\rho_{S_i}^p} \textbar{} PV\_p) \$.
\item
  Consumer node \(c\) creates the service instance \(s\) by excuting
  \(R_{S_i}\). \((id_s,\vec{d_s},\vec{\alpha_s})=D_{S_i}(\vec{r_s})\),
  which preprocesses \(c\)'s raw input \(\vec{r_s}\) to encrypt
  sensitive data, standalize the format to generate \(\vec{d_s}\).
  \(id_s\) is \(s\)'s universal id, \(\vec{\alpha_s}\) is the answer
  vector, which can be used for evaluation later.
\item
  \(c\) then calculate \$(Vol\_s, h\_e, \{PU\_\{l\_p\}\}) =
  O\_\{S\_i\}(\vec{d_s}) \$, where \(Vol_s\) is the common estimation of
  \(s\)'s volume, \(h_v\)(a.k.a. \textbf{Verification Height}) is the
  block height deadline for evaluation, and \(\{PU_{l_p}\}\) is a list
  (length \(l_p\)) of recommended provider's address selected by an
  algorithm powered by a pseudo-randomizer with the seed generated by
  \(\vec{d_s}\). In most cases, \(O_{S_i}\)create a list \(L_{P}^{S_i}\)
  with the addresses of all the providers of \(S_i\) by filtering the
  nodes with \(Cap_{Si} \gt 0\), then iterate every provider \(p_j\)
  with following steps:
\end{enumerate}

\begin{itemize}
\item
  Filter out \(p_j\) if it's available capacity is less than \(Vol_s\).
  Concretely, say \(p_j\) is already working on \(l_j\) instances of
  \(S_i\) with the volumes of \(\{Vol_{l_j}\}\), filter it out if \$
  Cap\^{}\{p\_j\}\emph{\{Si\} - \Sigma Vol}\{k\} \lt Vols \$
\item
  The fee \(p_j​\) required for task \(s​\) is calculated by
  \(f(s,p_j) = Rate_{S_i}^{p_j} \cdot Vol_s​\). Filter out \(p_j​\) if
  \(f(s,p_j) \gt _c​\),b since \(c​\) doesn't have enough balance to pay
  \(p_j​\) to run \(s​\).
\item
  one straightforward practise is to pick \(\{p_j\}\) where \(f(s,p_j)\)
  has the minimum values (The cheapest ones). Actually, there are other
  tactics to make more reasonable \(O_{S_i}\) utilizing other
  information recorded on chain. \emph{e.g.} taking the evaluation
  result from the blockchain of \(p_j\)'s last 10 services into account,
  or weight \(p_j\)'s experience history, etc.
\item
  \(O_{S_i}\) must utilize a pseudo-randomizer, otherwise it will be
  easy for the hackers to launch DDoS attack to \(p\). If the developer
  failed to follow this practise, the service is vulnerable and no
  provider would mount it.
\end{itemize}

After all these, \(c\) boardcasts the orientation message
\(M_{O}(PU_c,id_s, \vec{d_s}, \vec{\alpha_s}|PV_c)\). Other nodes
including selected providers (such as \(p\)) will get to know \(s\) is
assigned by calculating \$O\_\{S\_i\}(\vec{d_s}) \$.

To verify \(M_{O}\) during confirmation, bookkeeper nodes should:

\begin{itemize}
\tightlist
\item
  Check the signiture. If invalid, return false.
\item
  Run \$(Vol\_s, \{PU\_p\}, h\_e) = O\_\{S\_i\}(\vec{d_s}) \$, check if
  \(p\) has enough available capacity. If not, return false.
\item
  Check if \(c\) has enough balance. if not, return false.
\item
  Check if \(\vec{d_s}\) is valid. This is a service specific process.
\end{itemize}

If everything goes right, \(M_{O}\) is recorded into ledger (Say on the
height \(h_0\)). \(f(s,p)\) is frozen from \(c\)'s account immediately.

\begin{enumerate}
\def\labelenumi{\arabic{enumi}.}
\setcounter{enumi}{3}
\item
  Then it comes to "B" phase. \(p\) works on the mission by whatever
  practise, out of the topic' scope here.
\item
  \(p\) is supposed to give out a result vector \(\vec{z_s}\) before
  \(h_0+h_e\) reached. \(p\) can evaluate it's temporary \(\vec{\zeta}\)
  anytime by running
  \(({\upsilon}_s, c_{p_s},c_{a_s} )= E_{S_i}(\vec{d_s},\vec{\zeta}, \vec{\alpha_s} )\),
  where \({\upsilon}_s \in [0,1]\) represents the score of \(p\)'s
  service. \(c_{p_s}\) and \(c_{p_s}\) are the presumed service fee
  deposit to \(p\) and \(a_i\) 's account. The author \(a_i\) can design
  \(E_{S_i}\) to support boolean output 0 and 1 to represent only the
  status of success and failure.
\end{enumerate}

There's no fixed function to restrict what should \(p\) process to get
\(\vec{z_s}\), It's free for \(p\) to leverge any offchain solution or
resource on \(s\). But since lack of knowledge of \(\vec{r}\), \(p\)
cannot leverage \(R_{S_i}\).

Once \(p\) thinks \(\vec{z}\) is OK, it boardcasts the message
\$M\_\{E\}(id\_s, \vec{z}\textbar{}PV\_p) \$.

Then bookkeepers should check

\begin{itemize}
\tightlist
\item
  if latest block height is lower than \(h_0+ h_e\). If yes, record the
  message into ledger. Otherwise just ignore it.
\item
  if 51\% \(l_{p}\) of the selected providers return same \(\vec{z}\).
\end{itemize}

By calculating \$(v\_s, c\_\{p\_s\},c\_\{a\_s\}
)=E\_\{S\_i\}(\vec{d_s},\vec{z_s}, \vec{\alpha_s} ) \$, the amount of
\(F(s,p)\) is unfrozen in \(c\)'s account, and \$ c\_\{p\_s\}\$ and
\(c_{a_s}\) are transfered to \(p\) and \(a_i\) respectively.

The whole life cycle of \(s\) is finished.

\subsection{3. Solutions of some common
scenarios}\label{solutions-of-some-common-scenarios}

\subsubsection{3.1 Large input}\label{large-input}

One common challenge is that many types of services requires the
consumer to pass large size data to provider to process. Distributed
ledger network with the nature of being a "ledger", is neither expected
nor feasible to keep large size data. Some off-chain assisted mechanism
should be introduced to insure such data is successfully transferred.

Say service \(S\) has \(\vec{r} = \{\vec{\gamma},\vec{\tau}\}\) where
each element of \(\vec{\tau}\) contains large size data, while each
element of \(\vec{\gamma}\) has small size data. \(S\) should:

\begin{enumerate}
\def\labelenumi{\arabic{enumi}.}
\tightlist
\item
  Design \(D_S\) and \(N_S\) to calculate \(Vol_s\) irrelative with
  \(\vec{\tau}\). If \(Vol_s\) is related with some features of
  \(\vec{\tau}\) (say the size), just put the feature values as elements
  of \(\vec{\gamma}\)
\item
  \(S\) is better not to support manually provider pick up to avoid
  DDoS.
\item
  Providers should put it's off-chain server addresses into
  \$\vec{\rho_{S}} \$ in \(M_{provision}\).
\end{enumerate}

When consumer \(c\) is creating the new service instance \(s\), \(D_S\)
with input \(\vec{r}\) goes through following steps:

\begin{enumerate}
\def\labelenumi{\arabic{enumi}.}
\tightlist
\item
  use the same algorithm of \(N_s\) to calculate \(Vol_s\) and pick
  \(p\) with an algorithm pseudo-randomized by timestamp \(t\).\\
\item
  checks if \emph{SSK} (See Chapter 1) between \(p\) and \(c\) is
  already created. If not, create it.
\item
  encrypt \(\vec{\tau'} = E(\vec{\tau}, SSK(s,p))\). Post \$\vec{\tau'}
  \$ with \(p\)'s off-chain API, \(p\) returns \(id'_s\) to track \$
  \vec{\tau'} \$ . If error happens, go to step 1 and another provider
  will be picked.
\item
  Calculate \(\vec{d_s} = \{N'_s(\vec{\gamma}), id'_s, \epsilon, t\}\).
  \(N'_s\) is used to process small size parameter \(\vec{\gamma}\).
  \(\epsilon = CS(\vec{\tau'})\) is checksum.
\end{enumerate}

\(c\) sends \(M_{service}\) out. When bookkeepers verify
\(M_{service}\), they query \(p\)'s off-chain API with parameter
\(id_s'\) and its signiture, and get back \(\vec{\tau'}\), check if the
checksum matches with \(\epsilon\). Record the \(M_{service}\) into
blockchain then. If any error, just ignore it. The next bookkeepers will
try verify \(M_{service}\) until \(h_0+ h_e\) reached. If so, \(c\) can
reinitiate \(s\) again.

\(p\) only opens the GET API to the chain node (verified via
signatures), and limit the number of queries for particular \(id'_s\)
with the maxmium confirmations necessary (say 6 for Bitcoin and 12 for
Ethereum), except \(c\).

Since \(\vec{\tau'}\) is encrypted, bookkeepers can do nothing with it
except validation. \(p\) can decrypt the data and excute the service.

\subsubsection{3.2 Privacy with the service
provider}\label{privacy-with-the-service-provider}

From 3.1 we can see SSK can provide good protection from the parites
other than \(c\) and \(p\). But the question is: Is it possible to even
protect the privacy from \(p\)? The answer is yes if the author of the
service have a good design to split \(\vec{r}\) into subsets, create a
set of service instances, and none of the providers can see the whole
picture.

\subsubsection{3.3 Endurance}\label{endurance}

{[}TODO: Sharding verification. Order and answer already in
\(\vec{a}\){]}

Many services requires the providers to keep serving for a particular
long time. e.g. \(c\) wants to rent \(p\)'s storage to hold his video
for one whole month.

In this case, by defining extra parameter in \(E_S\), we can have
bookkeepers to

\begin{enumerate}
\def\labelenumi{\arabic{enumi}.}
\tightlist
\item
  Verify \(M_{resolve}\) whenever need to
\end{enumerate}

\subsection{4. Example Scenarios}\label{example-scenarios}

\#\#\# 4.1 Cloud mining

\(c\) wants to mine POW cryptocurrency \(\Gamma\), whose block cycle is
larger than that of our network. \(c\) finds there's a service called
\(S_{mine}\) to provide general mining service. So
\(\vec{r_s} = (\Gamma, B_h)\), while \(\Gamma\) is the symbol and
\(B_{h0}\) is the block content of height \(h\). \(D_{S_{mine}}\)
returns standard hash function request, and \(\vec{a}\) is the nounce.

\subsubsection{4.2 Gradient descent}\label{gradient-descent}

\(c\) is researching deep learning and has a neural network to train.
This is a typical Large Input scenario. \(c\) pass the encrypted NN
structure and training/testing set to \(p\) through offchain channel.


    % Add a bibliography block to the postdoc
    
    
    
    \end{document}
